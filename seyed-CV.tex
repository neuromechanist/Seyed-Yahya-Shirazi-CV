%%%%%%%%%%%%%%%%%%%%%%%%%%%%%%%%%%%%%%%%%
% Long Professional Curriculum Vitae
% LaTeX Template
% Version 1.1 (9/12/12)
%
% This template has been downloaded from:
% http://www.latextemplates.com
%
% Original author:
% Rensselaer Polytechnic Institute (http://www.rpi.edu/dept/arc/training/latex/resumes/)
%
% Important note:
% This template requires the res.cls file to be in the same directory as the
% .tex file. The res.cls file provides the resume style used for structuring the
% document.
%
%%%%%%%%%%%%%%%%%%%%%%%%%%%%%%%%%%%%%%%%%

%----------------------------------------------------------------------------------------
%	PACKAGES AND OTHER DOCUMENT CONFIGURATIONS
%----------------------------------------------------------------------------------------

\documentclass[10pt]{res} % Use the res.cls style, the font size can be changed to 11pt or 12pt here
\newcommand{\tss}[1]{\textsuperscript{#1}}
\usepackage{helvet} % Default font is the helvetica postscript font
% \usepackage{newcent} % To change the default font to the new century schoolbook postscript font uncomment this line and comment the one above
\usepackage{hyperref}
\usepackage[utf8]{inputenc}
\newsectionwidth{0pt} % Stops section indenting
\pagenumbering{roman}
\begin{document}

%----------------------------------------------------------------------------------------
%	YOUR NAME AND ADDRESS(ES) SECTION
%----------------------------------------------------------------------------------------

\name{Seyed Yahya Shirazi \\[4pt]} % Your name at the top

% If you don't want one of the addresses, simply remove all the text in the first or second \address{} bracket

\address{{\bf Ph.D. Candidate} \\ \bf{University of Central Florida} \\ email: \href{mailto:shirazi@ieee.org}{shirazi@ieee.org} \\ cell: +1 (407) 801-0090 \\ \href{https://neuromechanist.github.io}{https://neuromechanist.github.io}} % Your address 1

\address{} % Your address 2

%----------------------------------------------------------------------------------------

\begin{resume}

%----------------------------------------------------------------------------------------
%	OBJECTIVE SECTION
%----------------------------------------------------------------------------------------

% \section{\centerline{STATEMENT}}

% \vspace{8pt} % Gap between title and text

% A position in Personnel Administration utilizing skills in recruiting, training and compensation.\\ 

%----------------------------------------------------------------------------------------
%	EDUCATION SECTION
%----------------------------------------------------------------------------------------

\section{\centerline{EDUCATION}} 

\vspace{8pt} % Gap between title and text
{\sl Doctor of Philosophy}, Mechanical Engineering \hfill Jan. 2017 - Apr. 2021 \\ 
University of Central Florida (UCF), Orlando, FL \hfill (expected) \\ 
THESIS - Corticomuscular adaptation to mechanical perturbations in a seated locomotor task \\
GPA 3.75 

{\sl Master of Science}, Biomedical Engineering - Biomechanics \hfill Sep. 2011 - Feb. 2014 \\ 
Tehran Polytechnic, Tehran, Iran  \\ 
THESIS - Dynamic postural stability analysis on standing normal subjects and transtibial amputees \\
GPA 3.76 
 
{\sl Bachelor of Science}, Biomedical Engineering - Biomechanics  \hfill Sep. 2007 - Sep. 2011\\ 
Tehran Polytechnic, Tehran, Iran \\
GPA 3.70, Top 3, Honor Student
%----------------------------------------------------------------------------------------
\vspace{0.1in} % Some whitespace between sections
%----------------------------------------------------------------------------------------
%	PROFESSIONAL EXPERIENCE SECTION
%----------------------------------------------------------------------------------------
\section{\centerline{RESEARCH AND TEACHING EXPERIENCE}} 

\vspace{8pt} % Gap between title and text

{\sl University of Central Florida (UCF)}, Orlando , FL \hfill Jan. 2017 -- Present \\
\href{http://mae.ucf.edu/brain}{\underline{BRaIN}} Laboratory \hfill Graduate Research Assistant \\[2pt]
% \begin{itemize} \itemsep -2pt % Reduce space between items
% \item Developed new selection criteria for applicant screening . 
% \end{itemize}
{\bf Research focus:} Human neural and biomechanical responses to perturbation during locomotion. We
use a plethora of biomedical sensors (EEG, EMG, motion capture, and force sensors) to reach to an
integrative perspective about the human responses at cortical, muscular, and biomechanical levels. This project is supported by the National Institute on Aging of the National Institutes of Health (NIH). \\[2pt]
{\bf Secondary Focus:} Improving the sensor design and the processing
pipelines to overcome motion artifacts in biological signals. The new sensor design involves dual-layer EEG, where the second layer only records artifacts from the environment. The processing improvements include a subject-specific pipeline for detection and rejection of the artifacts.


{\sl University of Central Florida (UCF)}, Orlando , FL \hfill Jan. 2017 -- Aug. 2017 \\
\href{http://mae.ucf.edu/}{Mechanical and Aerospace Enginerring Department} \hfill Graduate Teaching Assistant \\[2pt]
I was a graduate teaching assistant for undergraduate solid mechanics (Spring 2017) and undergradute dynamics (Summer 2017). Both instructors gave me the highest possible evaluation, and one even wrote that I was their top TA during their 30 years of teaching experience.

{\sl Tehran Polytechnic}, Tehran, Iran \hfill Sept. 2011 -- Feb. 2014 \\ Department of Biomedical Engineering \hfill Graduate Research Assistant\\[2pt]
We compared responses of transtibial amputees and healthy participants to different mechanical perturbations during standing. I designed and created a multi-directional moving platform to perturb subjects standing on top of the platform.

{\sl Tehran Polytechnic}, Tehran, Iran \hfill Sept. 2013 -- May 2013 \\ Department of Biomedical Engineering \hfill Graduate Teaching Assistant\\[2pt]
I was teaching assitant for the Biomechaincs undergtraduate course delivered by Professor Mohamad Parnianpour.

% \vspace{-6pt} % Reduce space between positions at the same organization
{\sl Shahid Beheshti University of Medical Sciences (SBMU)}, Tehran, Iran \hfill March 2011 -- Aug. 2011 \\
Functional Neurosurgery and Stereotaxy Research Centre \hfill Biomechanics Research Intern 
\begin{itemize} \itemsep -2pt % Reduce space between items
\item Six months of first-hand neurosurgery OR experience 
\item Human spine 3D modeling from CAT scans using Mimics, Geomagic XOR, and SolidWorks
\item FEA analysis of the spinal cord disorders using Ansys
\end{itemize} 

{\sl Tehran Polytechnic}, Tehran, Iran \hfill Sept. 2009 -- May 2010 \\ Department of Biomedical Engineering \hfill Undergraduate Teaching Assistant\\[2pt]
I was the youngest teacher assistant in the deprtment for the undergraduate programming course for two semsters.

%----------------------------------------------------------------------------------------

\vspace{0.01in} % Some whitespace between sections

%----------------------------------------------------------------------------------------

\vspace{0.01in} % Some whitespace between sections

%----------------------------------------------------------------------------------------
%	PUBLICATIONS SECTION
%----------------------------------------------------------------------------------------

\section{\centerline{PUBLICATIONS (journals and proceedings)}} 

\vspace{8pt} % Gap between title and text
\textbf{Shirazi, S.Y.} and Huang, H. J. \textit{Differential theta-band signatures of the anterior cingulate and motor cortices during seated locomotor perturbations}, IEEE Transactions on Neural Systems and Rehabilitation Engineering, 2020 (under review)

{\bf Shirazi, S.Y.} and Huang H.J., {\sl More Reliable EEG Electrode Digitizing Methods Can Reduce Source Estimation Uncertainty, But Current Methods Already Accurately Identify Brodmann Areas}, Frontiers in Neuroscience, Nov 2019. (DOI: {\href{https://www.frontiersin.org/articles/10.3389/fnins.2019.01159/}{10.3389/fnins.2019.01159}})

{\bf Shirazi, S.Y.} and Huang H.J., {\sl Influence of Fiducial Mislocation on EEG Source Estimation}, Full Contribution Paper, IEEE/EMBS Conference on Neural Engineering (NER), San Francisco, CA, July 2019. (DOI: \href{https://doi.org/10.1109/NER.2019.8717065}{10.1109/NER.2019.8717065})

Sarafpour M., {\bf Shirazi S.Y.}, Shirazi E., Ghazaei F., and Parnianpour Z., {\sl Postural Balance Performance of Children with ADHD, with and without Medication: A Quantitative Approach}, Full Contribution Paper, 40\tss{th} Intl. Conference of the IEEE Engineering in Medicine and Biology Society (EMBC’18), Honolulu, HI, July 2018. (DOI: \href{https://ieeexplore.ieee.org/document/8512636}{10.1109/EMBC.2018.8512636})

Radaei F., {\bf Shirazi S.Y.}, Gharibzadeh S., Khashayar P, Ramezani M, and Fatouraee N: {\sl Evaluation of Relationship Between Balance Parameters and Bone Mineral Density}, 22\tss{nd} Iranian Conference on Biomedical Engineering (ICBME 2015), IOST, November 2015. (DOI: \href{https://ieeexplore.ieee.org/document/7404167}{10.1109/ICBME.2015.7404167})

{\bf Shirazi S.Y.}, Safaee Z., and Fatouraee N.: {\sl The Need for Stump-Socket Interface Pressure Measurement during Bidirectionally Perturbed Stance in Transtibial Amputees}, 21\tss{st} Iranian Conference on Biomedical Engineering (ICBME 2014), AUT, November 2014 (DOI: \href{https://ieeexplore.ieee.org/document/7043925}{10.1109/ICBME.2014.7043925})

\vspace{0.01in} % Some whitespace between sections
%----------------------------------------------------------------------------------------
%	Patent
%----------------------------------------------------------------------------------------

\section{\centerline{PATENT}} 
\vspace{8pt} % Gap between title and text
Shirazi S.Y., {\sl Centrifugal micro-viscometer: A lab-on-a-chip device to assess the viscosity of biological fluids}, Iran Patent \#77944, June 2012

\vspace{0.01in} % Some whitespace between sections

%----------------------------------------------------------------------------------------
%	conferences
%----------------------------------------------------------------------------------------

\section{\centerline{CONFERENCES (peer reviewed, selected)}} 

\vspace{8pt} % Gap between title and text
{\bf Shirazi, S.Y.} and Huang H.J., {\sl Use-dependent learning, not error-based learning, occurs during perturbed recumbent stepping}, 44\tss{th} Annual Meeting of the American Society of Biomechanics (ASB), Atlanta, Georgia, August 2020 \\[4pt]
{\bf Shirazi, S.Y.} and Huang H.J., {\sl Electrocortical and motor responses to perturbations are not necessarily coupled}, 4\tss{th} International Conference on Mobile Brain/Body Imaging (MoBI), La Jolla, CA, June 2020 (accepted, postponed to June 2021 due to the pandemic)\\[4pt]
Shaffer T., {\bf Shirazi S.Y.}, and Huang H.J., {\sl Older adults demonstrate sustained adaptation to frequent perturbations in recumbent stepping}, 49\tss{th} Society for Neuroscience Annual Meeting, Chicago, IL, October 2019 (ref: 763.18.2019)\\[4pt]
{\bf Shirazi S.Y.} and Huang H.J., {\sl Step Initiation Perturbations Lead to Sustained Adaptation}, 40\tss{th} Intl. Conference of the IEEE Engineering in Medicine and Biology Society (EMBC’18), Honolulu, HI, July 2018\\[4pt]
Huang H.J. and {\bf Shirazi S.Y.}, {\sl Adapting to Perturbations during Rhythmic Arm and Leg Movements}, 47\tss{th} Society for Neuroscience Annual Meeting, Washington, DC, November 2017 (ref: 2017-S-16447-SfN)\\[4pt]
{\bf Shirazi S.Y.}, Safaee, Z. and Fatouraee N.: Residuum-Prosthesis Interface Pressure Changes in Standing Transtibial Amputees during Bidirectional Surface Perturbations, 11th Iranian Congress of Orthotics and Prosthetics, IUMS, November 2014 \\[4pt]
Sarafpour M. and {\bf Shirazi S.Y.}, and Shirazi E.: Postural Stability Analysis of Attention Deficit/Hyperactive Adolescents, 1st National Conference on Individual-Social Empowerment of People with Special Needs, IAUQ, October 2014

\vspace{0.01in} % Some whitespace between sections

%----------------------------------------------------------------------------------------
%	talks
%----------------------------------------------------------------------------------------

\section{\centerline{INVITED TALKS}} 

\vspace{8pt} % Gap between title and text
{\bf Shirazi, S.Y.}, {\sl Motor learning from the neuromechanical perspective}, guest lecture for the Motor Learning and Development Course, EDPH 304, University of South Carolina Upstate, Spartanburg, SC, October 2020 \\[4pt]
{\bf Shirazi, S.Y.} and Huang H.J., {\sl Neuro-Rehabilitation}, Acada Talks, University of Central Florida, February 2018\\[4pt]
{\bf Shirazi S.Y.} Rahimi A., Fatouraee N., and Seddighi A.S., {\sl Stress Analysis in an Inter-Body Graft under Dynamic Loading via Finite Element Method}, Invited talk to the Seminar on Biomechanics of Spinal Column, SBMU, March 2011\\[4pt]

\vspace{0.01in} % Some whitespace between sections
%----------------------------------------------------------------------------------------
%	Outreach and Service
%----------------------------------------------------------------------------------------

\section{\centerline{OUTREACH and SERVICE}} 
\vspace{8pt} % Gap between title and text
{\bf Serivce}\\[4pt]
Version Control and GitHub workshop group leader, Annual Meeting of the American Society of Biomechanics (ASB), August 2020 \\[6pt]
{\bf Graduate mentorship}\\[4pt]
Co-advised a Masters student in Biomedical Engineering for the pre and post-operation postural stability of paitients with lower back pain, Advisor: Dr. Nasser Fatouraee, 2016\\[2pt]
Co-advised a Masters student in Clinical Psychology for the postural stability analysis on ADHD children, Advisor: Dr. Elham Shirazi, MD, 2015\\[6pt]
{\bf Undergraduate mentorship}\\[4pt]
Co-mentored a UCF female freshman to present at the Society of Neuroscience Meeting, Chicago, IL, 2019 \\[2pt]
Co-mentored a UCF female sophomore to prototype a magnetic break for assistive devices, BRaIN Lab, 2018\\[2pt]
Co-advised a Tehran Polytehcnic senior for her bachelor project on the corrleation between osteoporosis and postural stability, 2015\\[5pt]
{\bf Outreach}\\[4pt]
PedsAcademy STEM Day for the in-patient children, Nemours Children’s Hospital, May 2019 \\[2pt]
UCF STEM Day, demonstrated Biomechanics of the Muscles movement to middle-school students Oct 2018 \\[2pt]
UCF STEM Day, an advocate for “Staying active and smartphoning it!!” to K-12 students Oct 2017 \\[2pt]
Camp Connect II, lab members led a session called “Neuromechanics” to high school students June 2017
\vspace{0.01in} % Some whitespace between sections

%----------------------------------------------------------------------------------------
%	Professional Training
%----------------------------------------------------------------------------------------
\section{\centerline{EXTRACURRICULAR TRAINING }}
\vspace{15pt} % Gap between title and text
\begin{itemize} \itemsep -2pt
\item ISO 9001:2008 quality system management auditor, IMQ Academy
% \item ISO/TS 16949 internal auditor for automotive quality management system, IMQ Academy
\item ISO 13485:2003 and Legal requirements for medical device manufacturers and distributors, Iran's FDA
\item MATLAB Academy courses on programming techniques, data visualization, machine learning, and deep learning
\end{itemize}

%----------------------------------------------------------------------------------------
%	Awards
%----------------------------------------------------------------------------------------

\section{\centerline{AWARDS}}
\vspace{15pt} % Gap between title and text
\begin{itemize} \itemsep -2pt % Reduce space between items
\item Frank Hubbard Endowed Scholarship, UCF, 2020 
\item Honor graduate student at Tehran Polytechnic, 2011
\item Top 0.5\% National University Entrance Exams, 2007
\end{itemize} 

\end{resume} 
\end{document}